Nesta dissertação, analisamos a reputação de veículos de publicação e programas de pós-graduação em Ciência da Computação (CC) com foco em suas sub-áreas. Para realizar esta tarefa, consideramos as 37 sub-áreas em CC definidas pela Microsoft Academic Research e estendemos uma métrica de reputação baseada em redes de Markov, denominada P-score (\textit{Publication Score}).
%
Mais especificamente, examinamos o impacto obtido na reputação de conferências, periódicos e programas de pós-graduação no Brasil e nos Estados Unidos (EUA) em CC, ao considerarmos suas sub-áreas.
%
Nossos experimentos sugerem que a metodologia proposta produz resultados melhores que métricas basedas em citações. Também apresentamos um panorama das direções de pesquisa atuais do Brasil e dos EUA, que seja, em quais sub-áreas estes países possuem mais trabalhos de destaque no momento.
%
Esta análise de reputação sob a perspectiva de sub-áreas fornece informações adicionais para administradores de universidades, diretores de agências de fomento a pesquisa e representantes do governo que precisam decidir como alocar recursos de pesquisa limitados. Por exemplo, em CC, sabemos que o volume de publicações científicas nos EUA é significantemente superior ao volume de publicações brasileiras. Porém, este trabalho mostra que as sub-áreas em CC em que cada país possui maior impacto científico são basicamente disjuntas.

\keywords{Busca Acadêmica, Reputação, Sub-áreas, P-score}

%In this dissertation, we study the reputation of publication venues and graduate programs in Computer Science (CS) with focus on its subareas. For that we adopt the 37 CS subareas defined by Microsoft Academic Research and extend the usability of a reputation metric based on Markov networks, called P-score (for Publication Score).
%%
%More specifically, we study the impact to the reputation of CS conferences, journals, and graduate programs in Brazil and US when subareas are taken into account. 
%%
%Our experiments suggest that the extended P-scores yield better results when compared with citation counts. We also present an overview of current research directions of Brazil and US, i.e. on which subareas they have the most prominent work nowadays. 
%%
%This analysis of reputation on a per subarea basis provides additional insights for university officials, funding agencies directors, and government officials who need to decide how to allocate limited research funds. For instance, it is known that the volume of US scientific publications in CS is significant superior to the volume of Brazilian CS research. However, this work shows that the CS subareas in which each country has major scientific impact are basically disjoint. 
%
%\keywords{Academic Search, Reputation, Subareas, P-score}