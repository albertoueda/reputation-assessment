%É a apresentação concisa dos pontos importantes do trabalho. O
%resumo deve ter no mínimo 150 e no máximo 500 palavras. Deve ser seguido, obrigatoria-
%mente, logo abaixo de sua apresentação, do termo em negrito “ ‘Palavras-chave:” seguido
%das palavras-chave representativas do conteúdo do trabalho. O resumo deve ser composto
%14por uma sequência de frases completas. A primeira frase deve ser significativa, explicando o
%tema principal do trabalho. Na redação, dar preferência ao uso da terceira pessoa do singular
%e do verbo na voz ativa. Não se utiliza referência bibliográfica em resumos

In this dissertation, we study the reputation of publication venues and graduate programs in Computer Science (CS) with focus on its subareas. For that we adopt the 37 CS subareas defined by Microsoft Academic Research and extend the usability of a reputation metric based on Markov networks, called P-score (for Publication Score).
%
More specifically, we study the impact to the reputation of CS conferences, journals, and graduate programs in Brazil and US when subareas are taken into account. 
%
Our experiments suggest that the extended P-scores yield better results when compared with citation counts. We also present an overview of current research directions of Brazil and US, i.e. on which subareas they have the most prominent work nowadays. 
%
This analysis of reputation on a per subarea basis provides additional insights for university officials, funding agencies directors, and government officials who need to decide how to allocate limited research funds. For instance, it is known that the volume of US scientific publications in CS is significantly superior to the volume of Brazilian CS research. However, this work shows that the CS subareas in which each country has major scientific impact are basically disjoint. 

\keywords{Academic Search, Reputation, Subareas, P-score}

%% Cartaz
%In this dissertation, we study the reputation of publication venues and graduate programs in Computer Science (CS) with focus on its subareas. For that we adopt the 37 CS subareas defined by Microsoft Academic Research and extend the usability of a reputation metric based on Markov networks, called P-score. More specifically, we study the impact to the reputation of CS conferences, journals, and graduate programs in Brazil and US when subareas are taken into account. Our experiments suggest that the extended P-scores yield better results when compared with citation counts. We also present an overview of current research directions of Brazil and US, i.e. on which subareas they have the most prominent work nowadays. This analysis of reputation on a per subarea basis provides additional insights for university officials, funding agencies directors, and government officials who need to decide how to allocate limited research funds.


%% ICTIR
%In this work, we study the reputation of venues and research groups in Computer Science with focus on its subareas. We adopt the 37 subareas defined by Microsoft Academic Research as starting point and focus on the Computer Science departments in the US. More specifically, we study the impact to the reputation of venues and research groups when subareas are taken into account. For that we extend the usability of a metric called P-Score (for Publication Score), proposed in the literature. The reason we adopted the metric P-Score in this work is because it can be computed without using citation information. P-scores rely on a Markov network which can be used to model relations among researchers and among researchers and the venues they publish in. We run several experiments in which we compare the reputation of venues and research groups per subarea. The results suggest that the extended P-scores yield better results when compared with citation counts. In that matter, the analysis of reputation on a per subarea basis yields additional insights into the reputation of venues and research groups that are useful when deciding how to allocate research funds. 
%
%% SPIRE
%In this work, we study the reputation of Brazilian graduate programs in Computer Science (CS) on a per subarea basis, in light of the top US graduate programs. For that we adopt the 37 CS subareas defined by Microsoft Academic Research and rely on a reputation metric based on Markov networks, called P-score. We aim at understanding (i) how the reputation of Brazilian graduate programs vary per subarea, and (ii) what are the differences between the current research directions of the top CS graduate programs in Brazil and US. Experimental results show that Brazilian graduate programs are concentrated in subareas such as  Computer Graphics, Information Retrieval and Parallel Computing. It is known that the volume of US scientific publications in CS is significant superior to the volume of Brazilian CS research. However, this work shows that the CS subareas in which each country has major scientific impact are basically disjoint. For instance, the four subareas of most impact in US graduate programs are Computer Vision, Database, Theoretical Computer Science and Machine Learning, whereas the Brazilian graduate programs' publications on major venues are mostly related to Parallel Computing, Information Retrieval, Computer Graphics, and World Wide Web.